\documentclass{article}
\usepackage{alltt}
\usepackage{color}
\usepackage[hmargin=1in,vmargin=.5in,letterpaper]{geometry}
\renewcommand{\rmdefault}{Sabon}
\newcommand{\tab}{\hspace*{2em}}


\def\allttfont{}
\newenvironment{xalltt}{\begingroup\def\{{\char`\{}\def\}{\char`\}}%
\def\_{\char`\_}\def\\{\char`\\}\begin{alltt}\allttfont}{\end{alltt}\endgroup}
\newenvironment{smalltt}{\begin{small}\begin{xalltt}}{\end{xalltt}\end{small}}
\def\_{\texttt{\char`\_}}
\frenchspacing
\pagestyle{empty}

\begin{document}

\begin{center}
\large CS 207 Paper Exercises 2 // Due F 3/9 in class or via email to kohler@seas\\
{\color{blue}
Solutions by Loren McGinnis
}
\end{center}

\paragraph{(1)} Solution:\\
{\color{blue}
AF(SubChooser) = $<S>$ where:\\
$|S| = \texttt{size\_}$,\\
$S = [s_0,\dots,s_{n-1}]$ where $\forall 0 \leq i < |S|, s_i == \textsc{TRUE} \iff  \texttt{value\_ \& (1 << i)}$\\
RI(SubChooser) = $0 \leq \texttt{size\_} < 64$
}
\paragraph{(2)} Solution:
{\color{blue}
\begin{xalltt}
/** Returns the value of s_\{@a i\}
 * @pre 0 <= @a i < |S| */
BitwiseSubsetChooser::operator[](int i) const;
\end{xalltt}
}

\paragraph{(3)} Solution:
{\color{blue}
\begin{xalltt}
BitwiseSubsetChooser * subset_; //the subset that this iterator belongs to;
uint64_t mask_; //the iterator points to s_i <==> mask_ == 1 << i
\end{xalltt}
}
\paragraph{(4)} Solution:
{\color{blue}
\begin{xalltt}
std::vector<uint64_t> value_;
int size_;
\end{xalltt}
}
\paragraph{(5)} Solution:
{\color{blue}
\begin{xalltt}
/** Erase the element pointed to by @a it
 * @pre @a it is in position i, where 0 <= i < size()
 * @param it an iterator for this vector
 * @post new size() == old size() - 1
 * @post for all 0 <= j < i, new (*this)[j] == old (*this)[j]
 * @post for all i <= j < new size(), new (*this)[j] == old (*this)[j+1]
 * @return valid iterator for position i of new (*this)
 */
iterator vector<T>::erase(iterator it);
\end{xalltt}
}

\paragraph{(6)} Solution:
{\color{blue}
\begin{xalltt}
template <typename C, typename P>
void erase_if(C &x, P pred) \{
  C::iterator it = x.begin();
  while (it != x.end()) \{
    if (pred(*it)) \{
      it = x.erase(it);
    \} else \{
      ++it;
    \}
  \}
\}
\end{xalltt}
}

\paragraph{(7)}  Solution:\\
{\color{blue}
The assumption here is that the only non-constant work that \texttt{T::erase()} does is call \texttt{value\_.erase()}, which is \texttt{O(value\_.size())}, and probably \texttt{O(T.size())}, assuming such a function exists.
}

\paragraph{(8)}  Solution:
{\color{blue}
\begin{xalltt}
U::iterator U::erase(U::iterator it) \{
  assert(it.i\_ < value\_.size());
  int last = value\_.back();
  value\_.pop\_back();
  if (value\_.size() > it.i\_) \{
    *it = last;
  \}
  //else it now points past-the-end
  return it;
\}
\end{xalltt}
\end{document}
